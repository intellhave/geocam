\documentclass[12pt]{article}%
\usepackage{amssymb}
\usepackage{geometry}
\usepackage{indentfirst}%
\usepackage{amsmath}%
\setcounter{MaxMatrixCols}{30}%
\usepackage{amsfonts}%
\usepackage{graphicx}
%TCIDATA{OutputFilter=latex2.dll}
%TCIDATA{Version=5.00.0.2606}
%TCIDATA{CSTFile=40 LaTeX article.cst}
%TCIDATA{Created=Friday, March 30, 2007 00:21:27}
%TCIDATA{LastRevised=Friday, June 05, 2009 01:27:43}
%TCIDATA{<META NAME="GraphicsSave" CONTENT="32">}
%TCIDATA{<META NAME="SaveForMode" CONTENT="1">}
%TCIDATA{BibliographyScheme=Manual}
%TCIDATA{<META NAME="DocumentShell" CONTENT="Standard LaTeX\Blank - Standard LaTeX Article">}
%TCIDATA{Language=American English}
\newtheorem{theorem}{Theorem}
\newtheorem{acknowledgement}[theorem]{Acknowledgement}
\newtheorem{algorithm}[theorem]{Algorithm}
\newtheorem{axiom}[theorem]{Axiom}
\newtheorem{case}[theorem]{Case}
\newtheorem{claim}[theorem]{Claim}
\newtheorem{conclusion}[theorem]{Conclusion}
\newtheorem{condition}[theorem]{Condition}
\newtheorem{conjecture}[theorem]{Conjecture}
\newtheorem{corollary}[theorem]{Corollary}
\newtheorem{criterion}[theorem]{Criterion}
\newtheorem{definition}[theorem]{Definition}
\newtheorem{example}[theorem]{Example}
\newtheorem{exercise}[theorem]{Exercise}
\newtheorem{lemma}[theorem]{Lemma}
\newtheorem{notation}[theorem]{Notation}
\newtheorem{problem}[theorem]{Problem}
\newtheorem{proposition}[theorem]{Proposition}
\newtheorem{remark}[theorem]{Remark}
\newtheorem{solution}[theorem]{Solution}
\newtheorem{summary}[theorem]{Summary}
\newenvironment{proof}[1][Proof]{\noindent\textbf{#1.} }{\ \rule{0.5em}{0.5em}}
\geometry{left=1in,right=1in,top=1in,bottom=1in}
\begin{document}
\section*{Type Function Name Here}

\subsection*{Introduction}

In this space provide a brief introduction to familiarize the reader with the
function listed above. \ 

\subsection*{Subsidiaries}

List all functions used by this function; list all variables (and types) used
by this function.

Functions:

Global Variables:

Local Variables:

\subsection*{Description}

Begin this section with a detailed description of the function. \ For simple
functions provide sufficient theory to define the function, otherwise outline
the theory and cite "calculations were performed in Mathematica..." etc. if
applicable. \ 

Conclude this section with an explanation of why this function exists. \ This
would include initial motivation for the creation of the function, as well as
all primary programs (functions) that utilize the function. \ A brief history
of the function can also be given if it serves to explain why the function
exists. \ 

\subsection*{Practicum}

In this section describe the arguments passed to the function and explicitly
write the function in the following format:

ExampleFunction ( vertex, edge1, edge2, double, integer,...)

Provide a short example of the use of this function. \ This should be written
in code or pseudo-code. \ 

\subsection*{Limitations}

Provide a description of the limitations of the function. \ It should be clear
what works and doesn't work about the function from reading this section. \ 

\subsection*{Testing}

Describe how the function was tested. \ 

\subsection*{Future Work}

In this section, describe what changes or increased functionality are desired
for this function. \ It may be helpful to address some of the items listed in
the "Limitations" section. 


\end{document}
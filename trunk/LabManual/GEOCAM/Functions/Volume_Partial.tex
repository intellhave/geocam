%TCIDATA{Version=5.00.0.2606}
%TCIDATA{LaTeXparent=1,1,functions.tex}
                      

\section*{\texttt{VolumePartial::VolumePartial}}

\subsection*{Function Prototype}

\texttt{double VolumePartial(Vertex v\_i, Tetra t)}

\subsection*{Key Words}

Volume, tetrahedron, vertex, radius, Cayley-Menger determinant, standard
form, geoquant.

\subsection*{Authors}

Daniel Champion

\subsection*{Introduction}

\texttt{VolumePartial} calculates the partial derivative of the volume of a
tetrahedron with respect to the logarithm of the radius of a vertex.

\subsection*{Subsidiaries}

\textbf{Functions:} \ 

\texttt{listDifference}

\texttt{listIntersection}

\texttt{Simplex::isAdjVertex}

\textbf{Global Variables:} \ radii, etas.

\textbf{Local Variables:}

\subsection*{Description}

The volume of a tetrahedron only depends on the lengths of its edges as
calculated from the Cayley-Menger determinant. \ Thus for a given
tetrahedron $t$, it's partial derivatives with respect to $\log $ radii will
vanish except for those radii corresponding to the vertices of $t$. \ The
function \texttt{isAdjVertex} of the simplex class determines this
condition. \ \texttt{VolumePartial} then proceeds with an initialization
procedure that labels the vertices and edges (radii and etas) in standard
form. \ Specifically, \texttt{Volume\_Partial} receives as inputs an integer 
\texttt{i} corresponding to a vertex (in the vertex table) which is labeled
vertex v1. \ The remaining vertices are labeled $v2,v3,v4$, and the edges $%
e12,e13,e14,e23,e24,e34$ are labeled preserving the structure implied by the
assignment of the vertices. \ The radii $r_{1}$, $r_{2}$,... and eta values $%
Eta_{12},Eta_{13},$... are assigned to the corresponding vertices and edges.
\ 

The formula for the partial derivative in terms of these standard form
variables was calculated in Mathematica using the Cayley-Menger determinant,
that is:%
\begin{equation*}
288V^{2}=\det\left[ 
\begin{array}{ccccc}
0 & 1 & 1 & 1 & 1 \\ 
1 & 0 & L_{12}^{2} & L_{13}^{2} & L_{14}^{2} \\ 
1 & L_{12}^{2} & 0 & L_{23}^{2} & L_{24}^{2} \\ 
1 & L_{13}^{2} & L_{23}^{2} & 0 & L_{34}^{2} \\ 
1 & L_{14}^{2} & L_{24}^{2} & L_{34}^{2} & 0%
\end{array}
\right] ,
\end{equation*}
where the lengths were determined from the radii and eta values using the
formula%
\begin{equation*}
L_{ij}^{2}=r_{i}^{2}+r_{j}^{2}+2r_{i}r_{j}Eta_{ij}.
\end{equation*}

The formula obtained from Mathematica was outputted into the C programming
language using the function CForm. \ 

This function was designed for use in the optimization of the
Einstein-Hilbert-Regge functional using Newton's method. \ In this procedure
the gradient of the EHR functional is needed which contains the partial
derivatives of the volume. \ 

\subsection*{Practicum}

Usage:

\texttt{VolumePartial(Vertex v\_i, Tetra t)}

The integer \texttt{i} corresponds to a vertex in the vertex table, that is%
\begin{equation*}
\text{\texttt{VolumePartial (v\_i, t)} }=\frac{\partial }{\partial \log r_{i}%
}Volume(t).
\end{equation*}

\subsection*{Limitations}

\texttt{VolumePartial} was designed to output the correct partial derivative
for any integer $i$ in the vertex table and any tetrahedron $t$ in the
triangulation. \ 

\subsection*{Revisions}

subversion 757, 7/7/09, \texttt{VolumePartial} created within
NewtonsMethod.cpp.

subversion 1055, 3/12/10, \texttt{VolumePartial} converted to a geoquant.

\subsection*{Testing}

The partial derivative of volume of several known tetrahedra were calculated
using \texttt{Volume\_Partial} and verified using Mathematica.

\subsection*{Future Work}

The procedure that initializes the tetrahedron into standard form should be
removed from this program and placed elsewhere. \ There are several
occurrences of this type of procedure that should be consolidated. \ 

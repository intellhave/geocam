%TCIDATA{Version=5.00.0.2606}
%TCIDATA{LaTeXparent=1,1,functions.tex}
                      

\section*{\texttt{EHR}}

\subsection*{Function Prototype}

\texttt{double Example\_Function (Vertex, Edge1, Edge2, double, int,...)}

\subsection*{Key Words}

Enter all key words to this function here.

\subsection*{Authors}

Type the primary authors here. Example:

Daniel "Cliff Jumper" Champion

\subsection*{Introduction}

In this space provide a brief introduction to familiarize the reader with
the function listed above. \ 

\subsection*{Subsidiaries}

List all functions used by this function; list all variables (and types)
used by this function. \ Indent to show hierarchy.

Functions:

\qquad Function1

\qquad Function2

\qquad\qquad Function2.1

\qquad\qquad Function2.2

\qquad\qquad\qquad Function2.3

Global Variables:

Local Variables:

\subsection*{Description}

Begin this section with a detailed description of the function. \ For simple
functions provide sufficient theory to define the function, otherwise
outline the theory and cite "calculations were performed in Mathematica..."
etc. if applicable. \ 

Conclude this section with an explanation of why this function exists. \
This would include initial motivation for the creation of the function, as
well as all primary programs (functions) that utilize the function. \ A
brief history of the function can also be given if it serves to explain why
the function exists. \ 

\subsection*{Practicum}

Place any and all practical information about the function in this section.
\ Provide a short example of the use of this function if appropriate. \ This
should be written in code or pseudo-code written in the format below:

\bigskip

\qquad\texttt{begin repeat;}

\qquad\qquad\texttt{result = n+5;}

\qquad\qquad\texttt{end if result \TEXTsymbol{>} 5;}

\qquad\qquad\texttt{n=n+1;}

\qquad\texttt{end repeat;}

\bigskip

\subsection*{Limitations}

Provide a description of the limitations of the function. \ It should be
clear what works and doesn't work about the function from reading this
section. \ 

\subsection*{Revisions}

List the major revisions to the function with dates and a one sentence
comment. \ Example:

subversion 617, 6/8/09, \texttt{Example\_Function} created with severe
limitations.

subversion 618, 6/9/09, \texttt{Example\_Function} was fully commented and
initial testing complete.

subversion 619, 6/10/09, \texttt{Example\_Function} was augmented to utilize
the Geometry class.

\subsection*{Testing}

Describe how the function was tested. \ Include dates and names of test
results if possible.

\subsection*{Future Work}

In this section, describe what changes or increased functionality are
desired for this function. \ It may be helpful to address some of the items
listed in the "Limitations" section.

%TCIDATA{Version=5.00.0.2606}
%TCIDATA{LaTeXparent=0,0,classes.tex}
                      

%%%%% BEGINNING OF DOCUMENT BODY %%%%%
% html: Beginning of file: `clean.html'
% DOCTYPE HTML PUBLIC "-//W3C//DTD HTML 4.01//EN"
%  This is a (PRE) block.  Make sure it's left aligned or your toc title will be off. 

\section*{\texttt{Approximator}}

\label{f0}

\subsection*{Key Words}

\begin{quotation}
curvature flow, differential equations, Euler's method, Runga Kutta
\end{quotation}

\subsection*{Authors}

\begin{itemize}
\item Joe Thomas

\item Alex Henniges
\end{itemize}

\subsection*{Introduction}

\begin{quotation}
The Approximator class runs a curvature flow using one of several methods.
The class itself is abstract and the method is chosen by the instantiating
object.
\end{quotation}

\subsection*{Subsidiaries}

\begin{quotation}
Functions:
\end{quotation}

\begin{itemize}
\item \mbox{$[$}Functions\#Approximator::run run\mbox{$]$}
\end{itemize}

\begin{quotation}
Sub-classes:
\end{quotation}

\begin{itemize}
\item EulerApprox

\item RungaApprox
\end{itemize}

\begin{quotation}
Public Variables:
\end{quotation}

\begin{itemize}
\item radiiHistory

\item curvHistory

\item areaHistory

\item volumeHistory
\end{itemize}

\subsection*{Description}

\begin{quotation}
The \texttt{Approximator} class is the shell for running a curvature flow.
The class provides the functionality to determine what system of
differential equations to use, how to perform a step, and even which values
to record for later use. The system is provided by the user at run-time and
is defined to be a function that takes in an empty array of \texttt{double}s
and fills the array with the values calculated in the system of equations.
The \texttt{Approximator} class is abstract with an abstract method \texttt{%
step}. A class that extends \texttt{Approximator} implements \texttt{step}
with the method of approximation to use (i.e Euler's method). Lastly, the 
\texttt{Approximator} stores values after each \texttt{step} of a flow
according to which values were requested at construction. Values include
radii, curvatures, areas, and volumes. These histories can then be accessed
directly from the \texttt{Approximator} object.
\end{quotation}

\subsection*{Constructor}

\begin{quotation}
The constructor takes in a function that defines a system of differential
equations and a string of characters representing what histories to record.
For the function to be a \texttt{sysdiffeq} it must not return a value and
its only parameter is an array of doubles that will be filled in with values
after the function completes. The history string must be nul-terminated and
consisting of only valid characters. The valid characters currently are:
\end{quotation}

\begin{itemize}
\item r - Record radii

\item 2 - Record two-dimensional curvatures

\item 3 - Record three-dimensional curvatures

\item a - Record areas

\item v - Record volumes
\end{itemize}

\begin{quotation}
One cannot list both two- and three-dimensional curvatures.{\small }
\end{quotation}

\begin{verbatim}
{\small   typedef void (*sysdiffeq)(double derivs[]);
}
{\small   Approximator(sysdiffeq funct, char* histories);
}
{\small   
}
\end{verbatim}

\subsection*{Practicum}

\begin{quotation}
This example will show how to run a Yamabe curvature flow on the pentachoron
using precision and accuracy bounds (see \mbox{$[$}Functions\#run run%
\mbox{$]$}) while recording radii, curvatures, and volumes. It will show how
to initialize the system and also how to print out results at the end.%
{\small }
\end{quotation}

\begin{verbatim}
{\small int main(int argc, char** argv) {
}
{\small    map<int, Vertex>::iterator vit;
}
{\small    map<int, Edge>::iterator eit;
}
{\small    map<int, Face>::iterator fit;
}
{\small    map<int, Tetra>::iterator tit;
}
{\small      
}
{\small    vector<int> edges;
}
{\small    vector<int> faces;
}
{\small    vector<int> tetras;
}
{\small     
}
{\small     
}
{\small    time_t start, end;
}
{\small    
}
{\small    // File to read in triangulation from.
}
{\small 
   char from[] = "./Triangulation Files/3D Manifolds/Lutz Format/pentachoron.txt";
}
{\small    // File to convert to proper format.
}
{\small    char to[] = "./Triangulation Files/manifold converted.txt";
}
{\small    // Convert, then read in triangulation.
}
{\small    make3DTriangulationFile(from, to);
}
{\small    read3DTriangulationFile(to);
}
 
{\small    int vertSize = Triangulation::vertexTable.size();
}
{\small    int edgeSize = Triangulation::edgeTable.size();
}
{\small    
}
{\small    // Set the radii
}
{\small    for(int i = 1; i <= vertSize; i++) {
}
{\small 
      Radius::At(Triangulation::vertexTable[i])->setValue(1 + (0.5 - i/5.0) );        
}
{\small    }
}
{\small    // Set the etas
}
{\small    for(int i = 1; i <= edgeSize; i++) {
}
{\small        Eta::At(Triangulation::edgeTable[i])->setValue(1.0);
}
{\small    }
}
 
{\small 
   // Construct an Approximator object that uses the Euler method and Yamabe flow while
}
{\small    // recording radii, 3D curvatures, and volumes.
}
{\small    Approximator *app = new EulerApprox((sysdiffeq) Yamabe, "r3v");
}
 
{\small 
   // Run the Yamabe flow with precision and accuracy bounds of 0.0001 and stepsize of 0.01
}
{\small    app->run(0.0001, 0.0001, 0.01);
}
 
{\small    // Print out radii, curvatures and volumes
}
{\small 
   printResultsStep("./Triangulation Files/ODE Result.txt", &(app->radiiHistory), &(app->curvHistory));
}
{\small 
   printResultsVolumes("./Triangulation Files/Volumes.txt", &(app->volumeHistory));
}
 
{\small    return 0;
}
{\small }
}
\end{verbatim}

% html: End of file: `clean.html'

%%%%% END OF DOCUMENT BODY %%%%%
% In the future, we might want to put some additional data here, such
% as when the documentation was converted from wiki to TeX.
%

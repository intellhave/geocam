%TCIDATA{Version=5.00.0.2606}
%TCIDATA{LaTeXparent=0,0,functions.tex}
                      

%%%%% BEGINNING OF DOCUMENT BODY %%%%%
% html: Beginning of file: `clean.html'
% DOCTYPE HTML PUBLIC "-//W3C//DTD HTML 4.01//EN"
%  This is a (PRE) block.  Make sure it's left aligned or your toc title will be off. 

\section*{\texttt{printResultsNumSteps}}

\label{f0}

\begin{quotation}
{\small }
\end{quotation}

\begin{verbatim}
{\small 
   void printResultsNumSteps(char* fileName, vector<double>* radii, vector<double>* curvs)
}
{\small    
}
\end{verbatim}

\subsection*{Key Words}

\begin{quotation}
radii, curvatures, file, flow, vertex, print
\end{quotation}

\subsection*{Authors}

\begin{quotation}
Alex Henniges
\end{quotation}

\subsection*{Introduction}

\begin{quotation}
The \texttt{printResultsNumSteps} function prints out the results of a
curvature flow, with the results grouped by each step of the triangulation
but without labels. This format is used for with the GUI to create a
polygonal representation of curvatures. These results will be written to the
file given by \texttt{filename}.
\end{quotation}

\subsection*{Subsidiaries}

\begin{quotation}
Functions:

Global Variables:

Local Variables: \texttt{int vertSize}, \texttt{int numSteps}
\end{quotation}

\subsection*{Description}

\begin{quotation}
Prints the results of a curvature flow into the file given by \texttt{%
filename}. The results are curvature divided by radii values, and are given
by vectors of doubles. Most commonly, these vectors are taken from the
Approximator class after the flow is run. The \texttt{printResultsNumSteps}
function determines the number of vertices of the current triangulation and
the total number of steps are then derived from this and the size of the
vectors.

There are several ways to display the results. The \texttt{%
printResultsNumSteps} function groups by step but provides no labels and
does not print out radii, but instead curvautre divided by radii. The
purpose for this format is to create the ``Polygon flows'' in the GUI.
Therefore, it would be difficult for a human to read, but allows the
computer to do so much easier. An example is shown below.
\end{quotation}

\subsection*{Practicum}

\begin{quotation}
Example:{\small }
\end{quotation}

\begin{verbatim}
{\small 
  // Print the results of a curvature flow with Approximator app into file "ODEResult.txt"
}
{\small 
  printResultsNumSteps("./ODEResults.txt", app->radiiHistory, app->curvHistory);
}
{\small   
}
\end{verbatim}

\begin{quotation}
The output of such an example may then be{\small }
\end{quotation}

\begin{verbatim}
{\small          :
}
{\small          :
}
{\small       3.1425515910
}
{\small       3.1425294463
}
{\small       3.1425078130
}
 
{\small       3.1415926536
}
{\small       3.5987926375
}
{\small       3.5877016632
}
{\small       3.5768691746
}
{\small       3.5662905388
}
{\small          :
}
{\small          :
}
{\small   
}
\end{verbatim}

\subsection*{Limitations}

\begin{quotation}
Unlike the other print functions, the purpose of \texttt{printResultsNumSteps%
} is to only display the curvature divided by radii, and so is not limited
in the information it prints. On the otherhand, an overhaul of the entire
printing system would likely involve modifying this function.
\end{quotation}

\subsection*{Revisions}

\begin{itemize}
\item subversion 545, 9/29/08: Moved the printing of results out of calcFlow
and into a new function.

\item subversion 783, 6/18/09: Small modifications in response to changes in
the Approximator class.
\end{itemize}

\subsection*{Testing}

\begin{quotation}
The \texttt{printResultsNumSteps} function was tested by running multiple
curvature flows and printing the results. It was considered working when the
format of the data was as desired.
\end{quotation}

\subsection*{Future Work}

\begin{itemize}
\item 6/29 - Recreate the print functions to print more data and be more
flexible.

\item 6/29 - Move the print functions into the Approximator class.
\end{itemize}

% html: End of file: `clean.html'

%%%%% END OF DOCUMENT BODY %%%%%
% In the future, we might want to put some additional data here, such
% as when the documentation was converted from wiki to TeX.
%

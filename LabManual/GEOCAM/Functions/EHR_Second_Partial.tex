%TCIDATA{Version=5.00.0.2606}
%TCIDATA{LaTeXparent=1,1,functions.tex}
                      

\section*{\texttt{EHRSecondPartial::EHRSecondPartial}}

\subsection*{Function Prototype}

\texttt{double EHRSecondPartial (Vertex v\_i, Vertex v\_j)}

\subsection*{Key Words}

Einstein-Hilbert-Regge, functional, partial derivative, Hessian, geoquant.

\subsection*{Authors}

Daniel Champion

\subsection*{Introduction}

\texttt{EHRSecondPartial} calculates the second order partial derivatives of
the normalized Einstein-Hilbert-Regge functional with respect to log radii.
\ 

\subsection*{Subsidiaries}

\textbf{Functions:}

\qquad \texttt{CurvaturePartial}

\qquad\qquad\texttt{isAdjVertex}

\qquad \qquad \texttt{DualArea}

\qquad \qquad \qquad \texttt{DualAreaSegment}

\qquad \qquad \qquad \qquad \texttt{FaceHeight}

\qquad \qquad \qquad \qquad \qquad \texttt{EdgeHeight}

\qquad \qquad \qquad \qquad \qquad \qquad \texttt{PartialEdge}

\qquad\qquad\texttt{listDifference}

\qquad\qquad\texttt{listIntersection}

\qquad \texttt{TotalCurvature}

\qquad\qquad\texttt{Geometry::curvature}

\qquad \texttt{TotalVolume}

\qquad\qquad\texttt{Geometry::volume}

\qquad \texttt{VolumePartial}

\qquad\qquad\texttt{listDifference}

\qquad\qquad\texttt{listIntersection.}

\qquad \texttt{VolumeSecondPartial}

\textbf{Global Variables: }\ radii, etas.

\textbf{Local Variables:} \ none.

\subsection*{Description}

The normalized Einstein-Hilbert-Regge functional is given by the expression:%
\begin{equation*}
EHR=\frac{\sum\limits_{j}K_{j}}{\sqrt[3]{\sum\limits_{tetra\text{ }t}V_{t}}},
\end{equation*}
where $K_{i}$ is the curvature at vertex $j$, and $V_{t}$ is the volume of
tetrahedron $t$. \ It can be shown (see arXiv:0906.1560v1) that 
\begin{equation*}
\frac{\partial}{\partial\log r_{i}}\left( \sum\limits_{j}K_{j}\right) =K_{i},
\end{equation*}
hence the partial derivative of the normalized EHR functional simplifies to
become:%
\begin{equation*}
\frac{\partial}{\partial\log r_{i}}EHR=V^{-\frac{4}{3}}\left( K_{i}V-\frac {1%
}{3}K\sum\limits_{tetra\text{ }t}\frac{\partial V_{t}}{\partial\log r_{i}}%
\right) ,
\end{equation*}
where $V$ is the total volume of all tetrahedra in the triangulation and $K$
is the sum of the curvatures over all vertices in the triangulation. \
Differentiating this result with respect to $\log r_{j}$ yields:%
\begin{equation*}
\frac{\partial^{2}}{\partial\log r_{i}\partial\log r_{j}}EHR=V^{-\frac{4}{3}%
}\left( 
\begin{array}{c}
V\frac{\partial K_{i}}{\partial\log r_{j}}-\frac{1}{3}K_{i}\sum\limits_{t}%
\frac{\partial V_{t}}{\partial\log r_{j}}-\frac{1}{3}K_{j}\sum\limits_{t}%
\frac{\partial V_{t}}{\partial\log r_{i}} \\ 
+\frac{4}{9}KV^{-1}\sum\limits_{t}\frac{\partial V_{t}}{\partial\log r_{j}}%
\sum\limits_{t}\frac{\partial V_{t}}{\partial\log r_{i}}-\frac{1}{3}%
K\sum\limits_{t}\frac{\partial^{2}V_{t}}{\partial\log r_{i}\partial\log r_{j}%
}%
\end{array}
\right) .
\end{equation*}

When called, \texttt{EHRSecondPartial} calculates the formula above, that is:%
\begin{equation*}
\text{\texttt{EHR\_Second\_Partial (i,j)}}=\frac{\partial ^{2}}{\partial
\log r_{i}\partial \log r_{j}}EHR.
\end{equation*}

The use of this function is in the population of the Hessian matrix for the
normalized EHR functional. \ This Hessian matrix is used in the optimization
of the EHR functional using Newton's method.

\subsection*{Practicum}

As an example of the usage of \texttt{EHRSecondPartial}, the Hessian matrix
of the normalized EHR functional will be populated. \ In this example, the
Hessian matrix is the array EHRhessian. \ The example reduced computation
time by only calling \texttt{EHRSecondPartial} for the upper triangular
portion of the EHRhessian array, and symmetrically copies the entries above
the diagonal to the corresponding location below the diagonal. \
Furthermore, in C++ arrays begin indexing at zero, however the vertices of
the triangulations begin indexing at 1, requiring a shift of one in the
population step. \ Note that triangulations that do not label the vertices
consecutively will not be compatible with the following code. \ 

\bigskip

\qquad\texttt{double
EHRhessian[Triangulation::vertexTable.size()][Triangulation::vertexTable.size()];%
}

\qquad\texttt{for(int i = 0; i \TEXTsymbol{<}
Triangulation::vertexTable.size(); ++i) \{}

\qquad \qquad \texttt{for(int j = 0; j \TEXTsymbol{<}
Triangulation::vertexTable.size(); ++j) \{}

\qquad \qquad \qquad \texttt{Vertex vi = Triangulation::vertexYable[i+1];}

\qquad \qquad \qquad \texttt{Vertex vj = Triangulation::vertexTable[j+1];}

\qquad \qquad \qquad \texttt{if (i \TEXTsymbol{<}= j) \{}

\qquad \qquad \qquad \qquad \texttt{EHRhessian[i][j]=EHRSecondPartial( vi ,
vj );}

\qquad\qquad\qquad\qquad\texttt{EHRhessian[j][i]=EHRhessian[i][j];}

\qquad\qquad\qquad\qquad\texttt{\}}

\qquad\qquad\qquad\texttt{\}}

\qquad\qquad\texttt{\}}

\subsection*{Limitations}

\texttt{EHRSecondPartial} is fully operational with no known limitations. \
The function will output appropriate values provided it receives as inputs a
pair of integers in the vertex table. \ 

\subsection*{Revisions}

subversion 757, 7/7/09, \texttt{EHRSecondPartial} created.

subversion 1055, 3/12/10, \texttt{EHRSecondPartial}\ converted to a geoquant.

\subsection*{Testing}

This function has not been tested.

\subsection*{Future Work}

A testing regime should be instituted for this function. \ 

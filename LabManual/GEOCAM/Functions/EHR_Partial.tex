%TCIDATA{Version=5.00.0.2606}
%TCIDATA{LaTeXparent=1,1,functions.tex}
                      

\section*{\texttt{EHRPartial::EHRPartial}}

\subsection*{Function Prototype}

\texttt{double EHRPartial(int i)}

\subsection*{Key Words}

Einstein-Hilbert-Regge, functional, Newton's method, partial derivative,
geoquant.

\subsection*{Authors}

Daniel Champion

\subsection*{Introduction}

\texttt{EHRPartial} calculates the partial derivative of the normalized
Einstein-Hilbert-Regge functional with respect to log radii. \ 

\subsection*{Subsidiaries}

\textbf{Functions:}

\qquad \texttt{TotalVolume}

\qquad \texttt{TotalCurvature}

\qquad \texttt{VolumePartial}

\qquad\qquad\texttt{listDifference}

\qquad\qquad\texttt{listIntersection}

\textbf{Global Variables:} radii, etas, curvature, volume

\textbf{Local Variables:} \ none

\subsection*{Description}

The normalized Einstein-Hilbert-Regge functional is given by the expression:%
\begin{equation*}
EHR=\frac{\sum\limits_{j}K_{j}}{\sqrt[3]{\sum\limits_{tetra\text{ }t}V_{t}}},
\end{equation*}%
where $K_{i}$ is the curvature at vertex $j$, and $V_{t}$ is the volume of
tetrahedron $t$. \ It can be shown (see arXiv:0906.1560v1) that 
\begin{equation*}
\frac{\partial }{\partial \log r_{i}}\left( \sum\limits_{j}K_{j}\right)
=K_{i},
\end{equation*}%
hence the partial derivative of the normalized EHR functional becomes:%
\begin{align*}
\frac{\partial }{\partial \log r_{i}}EHR& =\frac{K_{i}\sqrt[3]{%
\sum\limits_{tetra\text{ }t}V_{t}}-\frac{1}{3}\left( \sum\limits_{tetra\text{
}t}V_{t}\right) ^{-\frac{2}{3}}\sum\limits_{tetra\text{ }t}\frac{\partial
V_{t}}{\partial \log r_{i}}\sum\limits_{j}K_{j}}{\left( \sum\limits_{tetra%
\text{ }t}V_{t}\right) ^{\frac{2}{3}}} \\
& =V^{-\frac{4}{3}}\left( K_{i}V-\frac{1}{3}K\sum\limits_{tetra\text{ }t}%
\frac{\partial V_{t}}{\partial \log r_{i}}\right) ,
\end{align*}%
where $V$ is the total volume of all tetrahedra in the triangulation and $K$
is the sum of the curvatures over all vertices in the triangulation. \ 
\texttt{EHRPartial (v\_i)} calculates $\frac{\partial }{\partial \log r_{i}}%
EHR$. \ 

The primary use of this function is in the calculation of the negative
gradient of the EHR functional for use in optimization of the functional
using Newton's method. \ The formula for $\frac{\partial }{\partial \log
r_{i}}EHR$ given above was also used in the calculation of the second order
partial derivatives of the EHR functional, implemented in \texttt{%
EHRSecondPartial}.

\subsection*{Practicum}

As an example of the use of this function, the calculation of the gradient
of the EHR functional will be calculated. \ The negative gradient will be
outputted as the array \texttt{EHRneg\_gradient}.

\bigskip

\qquad\texttt{double EHRneg\_gradient[Triangulation::vertexTable.size()];}

\qquad \texttt{for(int i=0; i \TEXTsymbol{<}
Triangulation::vertexTable.size(); ++i) \{}

\qquad \qquad \texttt{Vertex v\_i = Triangulation::vertexTable[i+1];}

\qquad \qquad \texttt{EHRneg\_gradient[i] = -1.0*EHRPartial(v\_i);}

\qquad\qquad\texttt{\}}

\subsection*{Limitations}

The function \texttt{EHRPartial} is fully functional with no known
limitations. \ It will return appropriate values so long as it is called
with an integer in the vertex table. \ 

\subsection*{Revisions}

List the major revisions to the function with dates and a one sentence
comment. \ Example:

subversion 757, 7/7/09, \texttt{EHRPartial} created.

subversion 1055, 3/12/10, \texttt{EHRPartial}\ converted to a geoquant.

\subsection*{Testing}

This function has not been tested.

\subsection*{Future Work}

A testing regime should be instituted for this function. \ 

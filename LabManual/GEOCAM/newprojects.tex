\documentclass{article}%
\usepackage{amsmath}%
\setcounter{MaxMatrixCols}{30}%
\usepackage{amsfonts}%
\usepackage{amssymb}%
\usepackage{graphicx}
%TCIDATA{OutputFilter=latex2.dll}
%TCIDATA{Version=5.00.0.2606}
%TCIDATA{CSTFile=40 LaTeX article.cst}
%TCIDATA{Created=Wednesday, October 07, 2009 14:46:09}
%TCIDATA{LastRevised=Wednesday, October 07, 2009 15:23:43}
%TCIDATA{<META NAME="GraphicsSave" CONTENT="32">}
%TCIDATA{<META NAME="SaveForMode" CONTENT="1">}
%TCIDATA{BibliographyScheme=Manual}
%TCIDATA{<META NAME="DocumentShell" CONTENT="Standard LaTeX\Blank - Standard LaTeX Article">}
\newtheorem{theorem}{Theorem}
\newtheorem{acknowledgement}[theorem]{Acknowledgement}
\newtheorem{algorithm}[theorem]{Algorithm}
\newtheorem{axiom}[theorem]{Axiom}
\newtheorem{case}[theorem]{Case}
\newtheorem{claim}[theorem]{Claim}
\newtheorem{conclusion}[theorem]{Conclusion}
\newtheorem{condition}[theorem]{Condition}
\newtheorem{conjecture}[theorem]{Conjecture}
\newtheorem{corollary}[theorem]{Corollary}
\newtheorem{criterion}[theorem]{Criterion}
\newtheorem{definition}[theorem]{Definition}
\newtheorem{example}[theorem]{Example}
\newtheorem{exercise}[theorem]{Exercise}
\newtheorem{lemma}[theorem]{Lemma}
\newtheorem{notation}[theorem]{Notation}
\newtheorem{problem}[theorem]{Problem}
\newtheorem{proposition}[theorem]{Proposition}
\newtheorem{remark}[theorem]{Remark}
\newtheorem{solution}[theorem]{Solution}
\newtheorem{summary}[theorem]{Summary}
\newenvironment{proof}[1][Proof]{\noindent\textbf{#1.} }{\ \rule{0.5em}{0.5em}}
\begin{document}
\section{Guidelines for starting new projects}

Keep in mind that the main files for the project are in some directory which
you created to download your subversion. Currently, we are primarily using
RapidSVN, found at http://rapidsvn.tigris.org/, though any subversion client
will do. The code is in google code, with address
https://geocam.googlecode.com/svn/trunk/. To install the subversion, you must
use the \textquotedblleft Checkout\textquotedblright\ command. We will assume
the directory . is the main directory the code has been downloaded into.

To start a new project:

\begin{enumerate}
\item Make a new directory under ./Projects. The name of the directory will
probably be the name of the project. Any files which are specifically used for
this project belong somewhere under this new directory. Files which have been
mostly tested and belong to the project as a whole will be in subdirectories
of the . directory.

\item Start a new project either with DevC++ or in unix (which means you must
maintain the makefile). The project will probably be the same name as the new
directory. Try to avoid using main.cpp as the name of the file with the main
function. 

\item Try not to use spaces in the file names, class names, etc.; instead, use
camelCase to separate multiple words. Avoid underscore, as it may be a problem
with interface with Mathematica. Be consistent with capitalizations, since
some operating systems distinguish capital letters and some do not.

\item You must adjust the include directories so that they can find the
necessary files. This can be done in DevC++ or in the Makefile. Try to use
relative paths when possible (such as ../../Geometry). In each .cpp and .h
file, includes should not have any paths in them. So, to find these files, the
project file and makefile must have the paths in it.

\item Style for commenting ???
\end{enumerate}

Keep in mind that the following files should be maintained in the repository:
.cpp, .h., .dev, Makefile, .tex, .txt. The following types of files should not
be in the repository (remove them if they find their way in: .o, .exe (except
for final code distributions), .aux, .bak, .aaa, .dvi




\end{document}
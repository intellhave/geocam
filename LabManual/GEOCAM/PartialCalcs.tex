
\documentclass[12pt]{article}
%%%%%%%%%%%%%%%%%%%%%%%%%%%%%%%%%%%%%%%%%%%%%%%%%%%%%%%%%%%%%%%%%%%%%%%%%%%%%%%%%%%%%%%%%%%%%%%%%%%%%%%%%%%%%%%%%%%%%%%%%%%%%%%%%%%%%%%%%%%%%%%%%%%%%%%%%%%%%%%%%%%%%%%%%%%%%%%%%%%%%%%%%%%%%%%%%%%%%%%%%%%%%%%%%%%%%%%%%%%%%%%%%%%%%%%%%%%%%%%%%%%%%%%%%%%%
\usepackage[pdftex]{graphicx}
\usepackage{multicol}
\usepackage{html,makeidx}
\usepackage{amsmath, amsthm}
\usepackage{amssymb}
\usepackage[subfigure]{ccaption}

\setcounter{MaxMatrixCols}{10}
%TCIDATA{OutputFilter=Latex.dll}
%TCIDATA{Version=5.00.0.2552}
%TCIDATA{<META NAME="SaveForMode" CONTENT="1">}
%TCIDATA{LastRevised=Monday, April 12, 2010 13:07:17}
%TCIDATA{<META NAME="GraphicsSave" CONTENT="32">}

\newtheorem{theorem}{Theorem}[section]
\newtheorem{lemma}[theorem]{Lemma}
\newtheorem{proposition}[theorem]{Proposition}
\newtheorem{corollary}[theorem]{Corollary}
\newtheorem{definition}[theorem]{Definition}
\newtheorem{conjecture}[theorem]{Conjecture}
\parindent 0in
\parskip 0.1in
\pagestyle{plain}

%\input{tcilatex}

\begin{document}

\title{The relationship between geometric quantities on a triangulation}
\author{Alex Henniges \\
%EndAName
\\
University of Arizona Undergraduate Research Program\\
Supervisor: Dr. David Glickenstein\\
}

\section{Summary}

This paper describes the calculations of partial derivatives of the
Normalized Einstein-Hilbert-Regge functional, hereby refered to as NEHR. The
purpose is to illustrate the the sort of calculations involved in studying
functionals across conformal classes, and in particular should prove the
correctness of the calculations being performed in the geocam project. 
\newline

\section{Notation}

We begin by providing some notation to help the reader. We will denote our
triangulation by $T$, and in general label simplices in index notation. A
vertex of T is represented as $\left\{ {i}\right\} $. An edge between
vertices $\left\{ {i}\right\} $ and $\left\{ {j}\right\} $ is denoted as $%
\left\{ {i,j}\right\} $ and similarly for faces and tetrahedra. When we are
referring to a geometric quantity defined on a particular simplex, we denote
it by its representative label and subscript the label with a simplex. For
example, curvatures have the label $K$, and thus the curvature at vertex $%
\left\{ {i}\right\} $ is $K_{i}$. Sometimes, a geometric quantity is defined
on a simplex contained within a simplex of higher dimension. A prime example
is a dihedral angle labeled by $\beta $. A dihedral angle is the angle of an
edge within a tetrahedron. In this case, we use a comma within or indexing
system to distinguish the smaller simplex with the larger one. Thus, the
dihedral angle at edge $\left\{ {i,j}\right\} $ with tetra $\left\{ {i,j,k,l}%
\right\} $ is $\beta _{ij,kl}$. We will define our labels once they appear
in the paper. \newline

\section{The NEHR Functional}

We start with the Einstein-Hilbert-Regge functional for triangulated
manifolds. The functional was chosen to closely mimic the Einstein-Hilbert
functional that is defined for smooth manifolds. In particular, we define
EHR as the sum of the curvatures,

\begin{equation}
EHR = \sum_{{i} \in T}{K_i}  \label{eq.EHR}
\end{equation}

The normalized form, NEHR, also mimics that of the NEHR for smooth surfaces.
In our case, it is

\begin{equation}
NEHR = \frac{\sum_{{i} \in T}{K_i}}{V^{1/3}}  \label{eq.NEHR}
\end{equation}

where $V$ is the total volume of the triangulation. Many of the
simplifications we do when calculating the partial derivatives will involve
substituting between \ref{eq.EHR} and \ref{eq.NEHR}.

We end this section with a very important key to our calcuations. The
following was derived by David Glickenstein in (ref paper):

\begin{equation}
\frac{\partial EHR}{\partial ln(r_i)} = K_i  \label{eq.EHR_Part}
\end{equation}

This states that the partial derivative of EHR with respect to the logarithm
of the radius of vertex ${i}$ is simply the curvature at vertex ${i}$. From
this point on we set $f_i = ln(r_i)$ and our derivatives are with respect to 
$f_i$. Switching between $f_i$ and $r_i$ can be done by a simple application
of the chain rule:

\begin{equation}
\frac{\partial G}{\partial ln(r_{i})}=\frac{\partial G}{\partial r_{i}}\frac{%
\partial r_{i}}{\partial ln(r_{i})}=r_{i}\frac{\partial G}{\partial r_{i}}
\label{eq.log_r}
\end{equation}

With this, we can begin calculating the partials.

\section{NEHR Partial}

We will first derivate the NEHR function with respect to the log of a
radius. This uses the product rule.

\begin{align}
\frac{\partial NEHR}{\partial f_{i}}& =K_{i}V^{-1/3}-\frac{1}{3}EHR\cdot
V^{-4/3}\frac{\partial V}{\partial f_{i}}  \notag \\
& =V^{-4/3}(K_{i}V-\frac{1}{3}r_{i}EHR\frac{\partial V}{\partial r_{i}})
\label{eq.NEHR_Part_f}
\end{align}

In our simplification, we make use of both a substitution from NEHR to EHR
and the property given in \ref{eq.log_r}. The derivative of the volume with
respect to radius $r_i$ is a long calculation that can be computed using the
equation given by the Cayley Menger Determinant to calculate volume.

We also want to take the derivative of NEHR with respect to the eta value of
an edge. There is no analog to equation \ref{eq.EHR_Part} and we don't take
logarithms, but the structure still appears similar. Again, the other
partials that result in this equation can be computed through large
calcualtions.

\begin{align}
\frac{\partial NEHR}{\partial \eta _{nm}}& =\sum_{i}{\frac{\partial K_{i}}{%
\partial \eta _{nm}}}V^{-1/3}-\frac{1}{3}EHR\cdot V^{-4/3}\frac{\partial V}{%
\partial \eta _{nm}}  \notag \\
& =V^{-4/3}\ast (\sum_{i}{\frac{\partial K_{i}}{\partial \eta _{nm}}}V-\frac{%
1}{3}EHR\frac{\partial V}{\partial \eta _{nm}})  \label{eq.NEHR_Part_eta}
\end{align}

\section{NEHR Second Partial}

We are also interested in the second derivative of NEHR with respect to
several variables. We first take the derivative of \ref{eq.NEHR_Part_f} with
respect to a second vertex, $f_j$.

\begin{align}
\frac{\partial ^{2}NEHR}{\partial f_{j}\partial f_{i}}& =\frac{\partial }{%
\partial f_{j}}\left[ K_{i}\cdot V^{-1/3}-\frac{1}{3}\cdot r_{i}\cdot
NEHR\cdot V^{-1}\cdot \frac{\partial V}{\partial r_{i}}\right]  \\
& =V^{-1/3}\cdot \frac{\partial K_{i}}{\partial f_{j}}-\frac{1}{3}\cdot
V^{-4/3}\cdot K_{i}\cdot \frac{\partial V}{\partial f_{j}}  \notag \\
& \ \ \ \ -\frac{1}{3}\cdot \frac{\partial r_{i}}{\partial f_{j}}NEHR\cdot
V^{-1}\cdot \frac{\partial V}{\partial r_{i}}  \notag \\
& \ \ \ \ -\frac{1}{3}\cdot r_{i}\cdot \frac{\partial EHR}{\partial f_{j}}%
\cdot V^{-4/3}\cdot \frac{\partial V}{\partial r_{i}}  \notag \\
& \ \ \ \ +\frac{4}{9}\cdot r_{i}\cdot \frac{EHR}{V^{1/3}}\cdot V^{-2}\cdot 
\frac{\partial V}{\partial f_{j}}\cdot \frac{\partial V}{\partial r_{i}} 
\notag \\
& \ \ \ \ -\frac{1}{3}\cdot r_{i}\cdot NEHR\cdot V^{-1}\frac{\partial ^{2}V}{%
\partial f_{j}\partial r_{i}}  \label{eq.NEHR_Part_f2_pre} \\
& =\frac{V^{-4/3}}{3}\Big[3\cdot V\cdot r_{j}\cdot \frac{\partial K_{i}}{%
\partial r_{j}}-r_{j}\cdot K_{i}\cdot \frac{\partial V}{\partial r_{j}} 
\notag \\
& \ \ \ \ -r_{j}\cdot \delta _{ij}\cdot EHR\cdot \frac{\partial V}{\partial
r_{i}}  \notag \\
& \ \ \ \ -r_{i}\cdot K_{j}\cdot \frac{\partial V}{\partial r_{i}}  \notag \\
& \ \ \ \ +\frac{4}{3}\cdot r_{i}\cdot r_{j}\cdot EHR\cdot V^{-1}\cdot \frac{%
\partial V}{\partial r_{j}}\cdot \frac{\partial V}{\partial r_{i}}  \notag \\
& \ \ \ \ -r_{i}\cdot r_{j}\cdot EHR\cdot \frac{\partial ^{2}V}{\partial
r_{j}\partial r_{i}}\Big]  \label{eq.NEHR_Part_f2}
\end{align}

We will attempt to point out as many simplifications as we can to prevent
confusion and doubt as to the correctness of the equation. In \ref%
{eq.NEHR_Part_f2_pre}, we choose to break down NEHR into EHR and an extra
volume term during the product rule. From \ref{eq.NEHR_Part_f2_pre} to \ref%
{eq.NEHR_Part_f2} we replace all of the appearances of $f_j$ with $r_j$. The 
$\delta_{ij}$ representing taking the derivative of $r_i$ with respect to $%
r_j$ and is 1 if $i=j$ and 0 otherwise.

We make no claims that the above formula is easy. Fortunately, the other two
second partials we will take mirror this equation very closely. Thus, one
can go through the steps of the above equation to convince themselves of all
three (plus, its great practice with taking derivatives.

\begin{align}
\frac{\partial^2 NEHR}{ \partial \eta_{nm} \partial f_i} &= \frac{\partial}{%
\partial \eta_{nm}}\Big[K_i \cdot V^{-1/3} - \frac{1}{3} \cdot r_i \cdot
NEHR \cdot V^{-1}\cdot \frac{\partial V}{\partial r_i}\Big] \\
&=V^{-1/3} \cdot \frac{\partial K_i}{\partial \eta_{nm}} - \frac{1}{3}\cdot
V^{-4/3} \cdot K_i \cdot \frac{\partial V}{\partial \eta_{nm}}  \notag \\
&\ \ \ \ - \frac{1}{3} \cdot r_i \cdot \frac{\partial EHR}{\partial \eta_{nm}%
} \cdot V^{-4/3} \frac{\partial V}{\partial r_i} + \frac{4}{9} \cdot r_i
\cdot \frac{EHR}{V^{1/3}} \cdot V^{-2} \cdot \frac{\partial V}{\partial
\eta_{nm}} \cdot \frac{\partial V}{\partial r_i}  \notag \\
&\ \ \ \ - \frac{1}{3} \cdot r_i\cdot NEHR \cdot V^{-1} \frac{\partial^2 V}{%
\partial \eta_{nm} \partial r_i}  \label{eq.NEHR_Part_f_eta_pre} \\
&= \frac{V^{-4/3}}{3} \Big[3 \cdot V \cdot \frac{\partial K_i}{\partial
\eta_{nm}} - \cdot K_i \cdot \frac{\partial V}{\partial \eta_{nm}}  \notag \\
&\ \ \ \ - r_i \cdot \sum_{j}{\frac{\partial K_j}{\partial \eta_{nm}}} \cdot 
\frac{\partial V}{\partial r_i} + \frac{4}{3} \cdot r_i \cdot EHR \cdot
V^{-1} \cdot \frac{\partial V}{\partial\eta_{nm}} \cdot \frac{\partial V}{%
\partial r_i}  \notag \\
&\ \ \ \ - r_i \cdot EHR \cdot \frac{\partial^2 V}{\partial \eta_{nm}
\partial r_i}\Big]  \label{eq.NEHR_Part_f_eta}
\end{align}

And $\frac{\partial ^{2}NEHR}{\partial \eta _{op}\partial \eta {nm}}$...

\begin{align}
\frac{\partial^2 NEHR}{ \partial \eta_{op} \partial \eta{nm} &= \frac{\partial}{\partial \eta_{op}}\Big[\sum_{i}{\frac{\partial K_i}{\partial\eta_{nm}}} * V^{-1/3} - \frac{1}{3} * EHR * V^{-4/3} * \frac{\partial V}{\partial \eta_{nm}}\Big] \\
						&=V^{-1/3} \cdot \frac{\partial K_i}{\partial \eta_{nm}} - \frac{1}{3}\cdot V^{-4/3} \cdot K_i \cdot \frac{\partial V}{\partial \eta_{nm}}\nonumber \\
						&\ \ \ \ - \frac{1}{3} \cdot r_i \cdot \frac{\partial EHR}{\partial \eta_{nm}} \cdot V^{-4/3} \frac{\partial V}{\partial r_i} + \frac{4}{9}\cdot r_i \cdot \frac{EHR}{V^{1/3}} \cdot V^{-2} \cdot \frac{\partial V}{\partial \eta_{nm}} \cdot \frac{\partial V}{\partial r_i}\nonumber \\
%						&\ \ \ \ - \frac{1}{3} \cdot r_i\cdot NEHR \cdot V^{-1} \frac{\partial^2 V}{\partial \eta_{nm} \partial r_i} \label{eq.NEHR_Part_eta2_pre}\\
%						&= \frac{V^{-4/3}}{3} \Big[3 \cdot  V  \cdot \frac{\partial K_i}{\partial \eta_{nm}} - \cdot  K_i \cdot  \frac{\partial V}{\partial \eta_{nm}} \nonumber \\
%						&\ \ \ \ - r_i \cdot \sum_{j}{\frac{\partial K_j}{\partial \eta_{nm}}} \cdot  \frac{\partial V}{\partial r_i} + \frac{4}{3} \cdot  r_i \cdot EHR \cdot V^{-1} \cdot  \frac{\partial V}{\partial\eta_{nm}} \cdot  \frac{\partial V}{\partial r_i} \nonumber \\
%						&\ \ \ \ - r_i \cdot  EHR \cdot \frac{\partial^2 V}{\partial \eta_{nm} \partial r_i}\Big]															
\label{eq.NEHR_Part_eta2}
\end{align}

\end{document}
